\documentclass[a4paper,12pt]{article}
%\documentclass[a4paper,10pt]{scrartcl}

\usepackage[utf8]{inputenc}
\usepackage{tablefootnote}
\usepackage[brazil]{babel}
\usepackage{subfig}
\usepackage{enumitem}
\usepackage{amssymb}
\usepackage{amsmath}
\newtheorem{theorem}{Teorema}
\usepackage{sectsty} % tamanho da fonte
\usepackage[font=small]{caption}
\usepackage{mathrsfs}             

\sectionfont{\fontsize{12}{15}\selectfont} % Tamanho da fonte das seções = 12
\subsectionfont{\fontsize{12}{15}\selectfont}
\linespread{1.5}
\usepackage[margin=2cm]{geometry}


\title{Algoritmo MiniMax com podas $\alpha$-$\beta$ utilizado no Gomoku}
\author{Fabio Moreira \and Marcello Klingelfus Junior}
\date{22 de Agosto de 2017}


\begin{document}
\maketitle

\begin{itemize}
 \item \textbf{Heurística e utilidade}
  
 A função heurística $H$($T$) deste trabalho pode ser definida como o somatório de todos os encadeamentos ao 
 quadrado que pertecem ao computador, multiplicado por um fator de bloqueio de jogada (detalhado adiante),
 subtraindo o somatório de todos os encadeamentos ao quadrado do adversário.
 A Equação~\ref{eq:fH} define matematicamente essa função.
 
 \begin{equation}\label{eq:fH}
   H(T) = \left(\sum_{i=1}^{N} (EC)^2\right)*TEB - \sum_{j=1}^{N} (EA)^2
 \end{equation}
 
 Onde:
 \begin{itemize}[label={--}]
  \item T: É uma determinada configuração do tabuleiro de \textit{gomoku}
  \item EC: Encadeamentos do computador
  \item EA: Encadeamentos do adversário
  \item TEB: Total de encadeamentos bloqueados, definido como a soma das peças bloqueadas do adversário ao adicionar uma peça pertencente ao computador.
  \item N: Total de encadeamentos do computador/adversário
 \end{itemize}
 
 Em uma partida normal, o adversário e o computador adicionam peças ao tabuleiro até que ou alguém faça uma sequência
 de cinco peças ou o tabuleiro seja totalmente preenchido, resultando em um empate. Para que se consiga atingir
 o objetivo do jogo, é necessário que as peças sejam agrupadas até que formem uma sequência de cinco peças. Para isso, é essencial
 que o computador entenda que peças agrupadas valem mais que peças ``soltas'' pelo tabuleiro. O objetivo de elevar os encadeamentos ao quadrado
 é justamente dizer ao computador que quanto maior o encadeamento, maiores serão as chances de atingir o objetivo buscado. O computador não joga sozinho, 
 assim, também é fundamental que ele compreenda que quanto mais encadeamentos grandes o adversário possuir, menores são as chances do computador
 de atingir seu objetivo, esse é o propósito da segunda parte da equação.
 O problema com essa heurística é que o adversário pode enganar o computador. Basta adicionar dois encadeamentos na mesma direção deixando um ``buraco'' entre eles.
 Com isso, o computador não verá essa jogada como uma ameaça e dará mais valor para outras jogadas, o fator TEB corrige essa deficiência.

 A função de utilidade $U(T)$ pode ser definida matematicamente através da Equação~\ref{eq:fU}. 
  \begin{equation}\label{eq:fU}
   H(T) = \left(\sum_{i=1}^{N} (EC)^2\right)*TEB - \left(\sum_{j=1}^{N} (EA)^2\right) * FJ
 \end{equation}
 O que difere as equações é o fator FJ (Fim de Jogo). Ao detectar o fim de jogo, seu valor é alterado de 1 para 100.
\end{itemize}

\begin{itemize}
 \item \textbf{Otimizações e estratégias}
 
 O tabuleiro foi projetado como uma matriz de tamanho $15 \times 15$. As peças são objetos que possuem 
 informações sobre: proprietário (computador, adversário ou vazio); seus vizinhos imediatos na vertical, horizontal 
 e diagonal e; sua posição no tabuleiro. Há também duas listas que guardam todas as peças utilizadas pelo 
 computador e adversário. Com esses dados, é possível otimizar o tempo de processamento evitando percorrer
 toda a matriz sempre que precisar recuperar informações sobre encadeamentos. Para o algoritmo MiniMax, a 
 árvore de estados será construída recursivamente ao longo da execução desse algoritmo. O tamanho do tabuleiro
 pode comprometer o desempenho do algoritmo. Assim, para a construção dos possíveis estados, será considerado
 um pedaço desse tabuleiro. Se o adversário adicionar uma peça fora desse espaço, o algoritmo passará
 a considerar um espaço que englobe essa peça.
 
\end{itemize}

\begin{itemize}
 \item \textbf{Detecção de fim de jogo e sequência de quatro peças}
 
 Ambas as detecções são feitas por um mesmo algoritmo. O algoritmo seqN (presente no código fonte) busca,
 a partir de um ponto aleatório, todos os encadeamentos presentes no jogo relacionados a um único jogador.
 Ao invés de percorrer todo o tabuleiro, o algoritmo vai ``empilhando'' peças na mesma direção. Quando o 
 encadeamento chega ao fim, sua profundidade é verificada e o encadeamento é guardado em uma lista.
 A posição da lista indica o tamanho do encadeamento, assim, ao encontrar uma profundidade de tamanho 3, o valor
 presente na posição 3 da lista é incrementado. Ao término do algoritmo, é só acessar essas posições e descobrir
 se há uma sequência de cinco peças indicando fim de jogo ou acessar as outras posições da lista para
 executar a heurística.
 
 
\end{itemize}


\end{document}
