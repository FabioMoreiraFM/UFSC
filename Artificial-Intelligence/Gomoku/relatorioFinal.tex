\documentclass[a4paper,12pt]{article}
%\documentclass[a4paper,10pt]{scrartcl}

\usepackage[utf8]{inputenc}
\usepackage{tablefootnote}
\usepackage[brazil]{babel}
\usepackage{subfig}
\usepackage{enumitem}
\usepackage{amssymb}
\usepackage{amsmath}
\newtheorem{theorem}{Teorema}
\usepackage{sectsty} % tamanho da fonte
\usepackage[font=small]{caption}
\usepackage{mathrsfs}             

\sectionfont{\fontsize{12}{15}\selectfont} % Tamanho da fonte das seções = 12
\subsectionfont{\fontsize{12}{15}\selectfont}
\linespread{1.5}
\usepackage[margin=2cm]{geometry}


\title{Algoritmo MiniMax com podas $\alpha$-$\beta$ utilizado no Gomoku}
\author{Fabio Moreira \and Marcello Klingelfus Junior}
\date{22 de Agosto de 2017}


\begin{document}
\maketitle

\begin{itemize}
 \item \textbf{Heurística e utilidade}
  
 A função heurística $H$($T$) deste trabalho pode ser definida como o somatório de todos os encadeamentos ao 
 quadrado que pertecem ao computador, multiplicado por um fator de bloqueio de jogada (detalhado adiante),
 subtraindo o somatório de todos os encadeamentos ao quadrado do adversário.
 A Equação~\ref{eq:fH} define matematicamente essa função.
 
 \begin{equation}\label{eq:fH}
   H(T) = \left(\sum_{i=1}^{N} (EC)^2\right)*TEB - \sum_{j=1}^{N} (EA)^2
 \end{equation}
 
 Onde:
 \begin{itemize}[label={--}]
  \item T: É uma determinada configuração do tabuleiro de \textit{gomoku}
  \item EC: Encadeamentos do computador
  \item EA: Encadeamentos do adversário
  \item TEB: Total de encadeamentos bloqueados, definido como a soma das peças bloqueadas do adversário ao adicionar uma peça pertencente ao computador.
  \item N: Total de encadeamentos do computador/adversário
 \end{itemize}
 
 Em uma partida normal, o adversário e o computador adicionam peças ao tabuleiro até que ou alguém faça uma sequência
 de cinco peças ou o tabuleiro seja totalmente preenchido, resultando em um empate. Para que se consiga atingir
 o objetivo do jogo, é necessário que as peças sejam agrupadas até que formem uma sequência de cinco peças. Para isso, é essencial
 que o computador entenda que peças agrupadas valem mais que peças ``soltas'' pelo tabuleiro. O objetivo de elevar os encadeamentos ao quadrado
 é justamente dizer ao computador que quanto maior o encadeamento, maiores serão as chances de atingir o objetivo buscado. O computador não joga sozinho, 
 assim, também é fundamental que ele compreenda que quanto mais encadeamentos grandes o adversário possuir, menores são as chances do computador
 de atingir seu objetivo, esse é o propósito da segunda parte da equação.
 O problema com essa heurística é que o adversário pode enganar o computador. Basta adicionar dois encadeamentos na mesma direção deixando um ``buraco'' entre eles.
 Com isso, o computador não verá essa jogada como uma ameaça e dará mais valor para outras jogadas, o fator TEB corrige essa deficiência.

 A função de utilidade $U(T)$ pode ser definida matematicamente através da Equação~\ref{eq:fU}. 
  \begin{equation}\label{eq:fU}
   H(T) = \left(\sum_{i=1}^{N} (EC)^2\right)*TEB - \left(\sum_{j=1}^{N} (EA)^2\right) * FJ
 \end{equation}
 O que difere as equações é o fator FJ (Fim de Jogo). Ao detectar o fim de jogo, seu valor é alterado de 1 para 1000.
\end{itemize}

\begin{itemize}
 \item \textbf{Otimizações e estratégias}
 
 O tabuleiro foi projetado como uma matriz de tamanho $15 \times 15$. As peças são objetos que possuem 
 informações sobre: proprietário (computador, adversário ou vazio), uma variável indicando se ele foi visitado ou não (sua utilidade será detalhada depois)
 e sua posição no tabuleiro.
 
 Há também uma lista que armazena o estado atual do jogo em relação aos encadeamentos existentes. A cada jogada, avalia-se
 o efeito da adição de uma peça (do computador ou do adversário) no tabuleiro. Com isso, evita-se analisar as cadeias toda vez que uma peça é adicionada ao jogo.
 
 Para o algoritmo MiniMax, a árvore de estados será construída recursivamente ao longo da execução desse algoritmo. O tamanho do tabuleiro
 pode comprometer o desempenho do algoritmo. Assim, para a construção dos possíveis estados, será considerado
 um pedaço desse tabuleiro. Se o adversário adicionar uma peça fora desse espaço, o algoritmo passará
 a considerar um espaço que englobe essa peça.
 
\end{itemize}

\begin{itemize}
 \item \textbf{Detecção de fim de jogo e sequência de quatro peças}
 
 A detecção não é realizada por um algoritmo específico, mas é extraído do resultado do método 'hu' (atualiza o encadeamento) que por sua vez chama o método 'Busca Pontual' (procura alterações).
 Esse método mantém a lista de encadeamentos do jogo sempre atualizada. Ao adicionar uma peça, o método verifica quantos encadeamentos 
 há ao redor daquela peça, por exemplo, se há um encadeamento de tamanho dois na vertical à esquerda e de tamanho um à direita, o método decrementa
 dois encadeamentos de tamanho dois da lista e incrementa um encadeamento de tamanho quatro. Quando o método finaliza a execução, 
 basta verificar se há um encadeamento de tamanho quatro ou cinco. Ao contrário do método imaginado inicialmente e detalhado
 no trabalho parcial, há um grande ganho de desempenho ao analisar somente os encadeamentos modificados ao adicionar uma nova peça ao jogo.

\end{itemize}

\begin{itemize}
  \item \textbf{Decisões de projeto}
  
  O trabalho foi dividido em três classes:
  
  \textbf{Controlador.py} - Inicia e controla o fluxo do jogo, chama os métodos das classes IA e Tabuleiro quando necessário.
  
  \textbf{IA.py} - Possui o algoritmos: MiniMax, atualização da lista de encadeamentos, heurística e demais ferramentas que auxiliam na
  implementação destes. Pelo fato de usar intensamente a lista de encadeamentos do tabuleiro, decidiu-se armazená-la nesta classe ao invés da classe 
  Tabuleiro.py, caso contrário acarretaria uma queda no desempenho ao ter que chamar uma função, colocar na pilha, retirar, etc. inúmeras vezes.
  
  \textbf{Tabuleiro.py} - Cria o tabuleiro, faz todas as inicializações e validações. Responsável por todas as alterações
  envolvendo o tabuleiro (adição/remoção de peças). Controla o ``subtabuleiro'' que será utilizado para o algoritmo MiniMax. 
 
\end{itemize}

\begin{itemize}
  \item \textbf{Limitações}
  
  - Para tamanhos de tabuleiro grandes, maiores que $11\times11$, a jogada do computador pode afetar o tempo
  de processamento da escolha da melhor jogada. Por exemplo, ao adicionar uma peça no início do tabuleiro, o valor de
  alpha será alto ao adicionar peças ao lado dessa e baixo ao adicionar peças longe dela. Dessa forma, o algoritmo
  miniMax irá realizar a poda mais cedo, acelerando o algoritmo. Ao contrário, se adicionar uma peça no fim, a poda ocorrerá mais tarde.
  
\end{itemize}

\begin{itemize}
  \item \textbf{Principais métodos}
  
  \textbf{IA.minimax()} - Implementa o algoritmo minimax com poda alpha e beta.
  
  \textbf{IA.BuscaPontual()} - Busca os vizinhos do último ponto inserido e os tamanhos dos seus encadeamentos 
  e faz a atualização da lista de encadeamentos.  
  
  \textbf{IA.heuristica()} - Retorna a pontuação heurística para um certo nó folha.
\end{itemize}
\end{document}
